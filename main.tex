\documentclass{article}
\usepackage[utf8]{inputenc}
\usepackage{amsthm,amsmath,amssymb,amsfonts}
% % Theorem environments.
 \newtheorem{theorem}{Theorem}
 \newtheorem{lemma}[theorem]{Lemma}
 \newtheorem{proposition}[theorem]{Proposition} 
 \newtheorem{corollary}[theorem]{Corollary}
 \newtheorem{fact}[theorem]{Fact} 
 \newtheorem{definition}[theorem]{Definition}
\usepackage{tikz}
\def\checkmark{\tikz\fill[scale=0.4](0,.35) -- (.25,0) -- (1,.7) -- (.25,.15) -- cycle;}
 

\title{Measure Theory Notes}
\author{jwweber }
\date{January 2023}

\begin{document}

\maketitle

\section{Introduction}
\begin{definition}
    A collection $\mathcal{F}$ of subsets of $\Omega$ is a $\sigma$-algebra if:
    \begin{enumerate}
        \item $\Omega \in \mathcal{F}$
        \item $A\in \mathcal{F}\implies A^c \in\mathcal{F}$
        \item $(A_n)\subset \mathcal{F}\implies \bigcup_{n=1}^{\infty}A_n\in\mathcal{F}$
    \end{enumerate}
\end{definition}

\begin{definition}
Let $(X,\Sigma)$,$(Y,T)$ be measurable spaces. A function $f:X\to Y$ is measurable if $\forall E\in T, f^{-1}(E):=\{x\in X:f(x)\in E\}\in \Sigma$.
\end{definition}
Example:\\ Let $X=Y=\{1,2,3\}$, \\$\Sigma=\{\{1,2,3\},\{1,2\},\{3\},\{2\},\{1,3\},\{2,3\},\{1\},\emptyset\}$ \\ $T=\{\{1,2,3\},\{2,3\},\{1\},\emptyset\}$ \\

Define $f(1)=3,f(2)=2,f(3)=1$\\
Consider each $E\in T$ \\
$E_1=\{1,2,3\},f(1)=3\in E_1,f(2)=2\in E_1,f(3)=1\in E_1$\\ $\implies f^{-1}(E_1)=\{1,2,3\}\in \Sigma$ \checkmark\\

$E_2=\{2,3\},f(1)=3\in E_2,f(2)=2\in E_2,f(3)=1\notin E_2$\\ $\implies f^{-1}(E_2)=\{1,2\}\in \Sigma$ \checkmark\\

$E_3=\{1\},f(1)=3\notin E_3,f(2)=2\notin E_3, f(3)=1\in E_3$\\ $\implies f^{-1}(E_3)=\{3\}\in \Sigma$ \checkmark\\

$E_4=\emptyset$\\ $\implies f^{-1}(E_4)=\emptyset\in \Sigma$ \checkmark\\

$\implies f$ is a measurable function. 

\begin{definition}
    Let $(\Omega, \mathcal{F})$ be a measurable space. Then the function $\mu:\mathcal{F}~\rightarrow~[0,\infty]$ is called a measure. 
\end{definition}
\end{document}
