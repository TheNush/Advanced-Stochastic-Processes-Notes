\documentclass[10pt, oneside]{article} 
\usepackage{amsmath, amsthm, amssymb, calrsfs, wasysym, verbatim, bbm, color, graphics, geometry}
\usepackage{hyperref}
\usepackage{ourmacros}

\geometry{tmargin=.75in, bmargin=.75in, lmargin=.75in, rmargin = .75in}  

\newcommand{\R}{\mathbb{R}}
\newcommand{\C}{\mathbb{C}}
\newcommand{\Z}{\mathbb{Z}}
\newcommand{\N}{\mathbb{N}}
\newcommand{\Q}{\mathbb{Q}}
\newcommand{\Cdot}{\boldsymbol{\cdot}}

\newtheorem{thm}{Theorem}
\newtheorem{defn}{Definition}
\newtheorem{conv}{Convention}
\newtheorem{rem}{Remark}
\newtheorem{lem}{Lemma}
\newtheorem{cor}{Corollary}


\title{Advanced Stochastic Processes Seminar Notes}
\author{AC, DG, JW, TC}

\begin{document}

\maketitle
\tableofcontents

\vspace{.25in}

\section{Motivations for Measure Theoretic Perspective}

\section{Introduction}
\begin{defn}[$\sigma$-Algebra]
    \label{defn:SA}
    Let $\mathcal{F}$ be a collection of subsets of $\Omega$, for some set $\Omega$.
    $\mathcal{F}$ is called a $\sigma$-Algebra if and only if:
    \begin{enumerate}
        \item $\Omega \in \mathcal{F}$, 
        \item for any $A \subseteq \Omega$, $A \in \mathcal{F} \implies 
            A^{c} \in \mathcal{F}$, 
        \item if $\{A_{i}\}_{i \in I}$ is a countable sequence of sets, where $A_i 
            \in \mathcal{F}$, for all $i \in I$, then $\bigcup_{i \in I}A_{i} \in 
            \mathcal{F}$.  
    \end{enumerate} 
\end{defn} 

Note that the intersection of an arbitrary family of $\sigma$-Algebras over the 
same set $\Omega$ is also a $\sigma$-Algebra over $\Omega$ (verify as an exercise). 

\begin{defn}[Generated $\sigma$-Algebra]
    \label{defn:gen-SA}
    Let $X$ be an arbitrary collection of subsets of $\Omega$. If $X$ is not a 
    $\sigma$-Algebra, we can recursively construct a $\sigma$-Algebra, denoted $[X]$,
    by adding subsets of $\Omega$ until all of requirements of the above definition
    are satisfied. $[X]$ is the smallest $\sigma$-Algebra of $\Omega$ that contains
    all elements of $X$. 
\end{defn}

The following example will clarify Definition~\ref{defn:gen-SA}.
\dhanush{Can someone add an example?} 

An important generated $\sigma$-Algebra is the $Borel \sigma-Algebra$, denoted 
$\mathcal{B}(\R)$, and is defined over the extended real line, $\R^{*} = [-\infty,
+\infty]$. It is generated by the set of all open sets in $\R^{*}$, i.e., 
$\mathcal{B}(\R) = [\R^{*}]$. 

\begin{defn}[Measurable Space]
    \label{defn:space}
    Given a set $\Omega$ and a $\sigma$-Algebra $\mathcal{F}$ over it, the ordered 
    pair $(\Omega,\mathcal{F})$ is called a measurable space. 
\end{defn}

\begin{defn}[Measurable Map]
    \label{defn:map}
    Let $(X,\mathcal{A})$ and $(Y,\mathcal{B})$ be two measurable spaces. Let 
    $f \colon X \to Y$ be a map. $f$ is said to be measurable if and only if 
    for every $B \in \mathcal{B}$, 
    \[f^{-1}(B) = \{x \in X \colon f(x) \in B\} \in \mathcal{A}.\]
\end{defn}

Note that there are two components that contribute to a map between measurable spaces
being measurable: (i) the choice of $\sigma$-Algebras $\mathcal{A}$ and $\mathcal{B}$,
and (ii) the map $f$ itself. 

\begin{defn}[Map-Induced $\sigma$-Algebra]
    \label{defn:map-ind-SA}
    Let $(X,\mathcal{A})$ be a measurable space and let f be the map $f \colon X 
    \to Y$. We then define the map-induced $\sigma$-Algebra to be the set 
    \[\mathcal{B}_f = \{B \in \mathbb{P}(Y) \colon f^{-1}(B) \in \mathcal{A}\}. \]
    It is trivial to verify that $\mathcal{B}_f$ is indeed a $\sigma$-Algebra of $Y$. 
\end{defn}

Note that $\mathcal{B}_f$ is the largest $\sigma$-Algebra over $Y$ for which the 
given map $f \colon X \to Y$ is measurable. This is because $\mathcal{B}_f$ is 
constructed by considering all possible subsets of $Y$ via the power set of $Y$, 
denoted in Definition~\ref{defn:map-ind-SA} as $\mathbb{P}(Y)$. 

\begin{lem}
    A map $f$ from a measurable space $(X,\mathcal{A})$ to $\R^{*}$ is said to be 
    measurable if and only if for all $c \in \R$, 
    \[\{x \in X \colon f(x) > c\} \in \mathcal{A}. \]
\end{lem}
\dhanush{This wasn't in the lectures but seemed like a good intro proof. }

\begin{proof}
    
\end{proof}

\subsection{Measure Theoretic Probability Theory}

\subsection{Intuition}

\section{Lebesque Measures}

\end{document}
